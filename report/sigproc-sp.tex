\documentclass{acm_proc_article-sp}

\begin{document}

\title{Search engine for {\ttlit Politics} Lovers}

% AUTHORS
\numberofauthors{2} 
\author{
\alignauthor
David Dias\titlenote{Master Student}\\
       \affaddr{IST, Technical University of Lisbon}\\
       \affaddr{Av. Prof. Doutor Aníbal Cavaco Silva}\\
       \affaddr{2744-016 Porto Salvo, Portugal}\\
       \email{david.dias@ist.utl.pt}
% 2nd. author
\alignauthor
Rui Francisco\titlenote{Master Student}\\
       \affaddr{Institute for Clarity in Documentation}\\
       \affaddr{Av. Prof. Doutor Aníbal Cavaco Silva}\\
       \affaddr{2744-016 Porto Salvo, Portugal}\\
       \email{rui.miguel.guerreiro.francisco@ist.utl.pt}
}
\maketitle
\begin{abstract}
Abstract Abstract Abstract Abstract Abstract Abstract Abstract Abstract Abstract Abstract Abstract Abstract Abstract Abstract Abstract Abstract Abstract Abstract Abstract Abstract Abstract Abstract Abstract Abstract Abstract Abstract Abstract Abstract Abstract Abstract Abstract Abstract Abstract Abstract Abstract Abstract 

\end{abstract}

\keywords{TD-IDF, BM25, Search, Crawler} % NOT required for Proceedings


\section{Introduction}
INTRODUCTION INTRODUCTION INTRODUCTION INTRODUCTION INTRODUCTION INTRODUCTION INTRODUCTION INTRODUCTION INTRODUCTION INTRODUCTION INTRODUCTION INTRODUCTION INTRODUCTION INTRODUCTION INTRODUCTION INTRODUCTION 


\section{Arquitecture}
ARQUITECTURE ARQUITECTURE ARQUITECTURE ARQUITECTURE ARQUITECTURE ARQUITECTURE ARQUITECTURE ARQUITECTURE ARQUITECTURE ARQUITECTURE ARQUITECTURE ARQUITECTURE ARQUITECTURE ARQUITECTURE ARQUITECTURE ARQUITECTURE 


\subsection{Mining scheme}
MINING SCHEME MINING SCHEME MINING SCHEME MINING SCHEME MINING SCHEME MINING SCHEME MINING SCHEME MINING SCHEME MINING SCHEME MINING SCHEME MINING SCHEME MINING SCHEME MINING SCHEME MINING SCHEME MINING SCHEME 

\subsection{Searching scheme}


\subsection{Components}

\subsubsection{NewsCollection and Storage}

\subsubsection{News Search}

\subsubsection{Extraction of Named Entities}
A classe que implementa esta funcionalidade é entityExtraction/getNamesOfPersonalities.py. 
A função que implementa esta funcionalidade e extractNames. 
Esta função obtém todas as noticias guardadas na base de dados. 
Depois pega em cada noticia e divide a por frases usando a função nltk.sent_tokenize da biblioteca nltk. 
Em seguida usando a função nltk.word_tokenize divide cada frase por palavras e logo em seguida classifica cada palavra utilizado a função nltk.pos_tag. 
Depois de todas as palavras estarem classificada construi  uma parse tree através da função nltk.ne_chunk.
Depois verifica se cada nó e do tipo PERSON caso seja, verifica se este se encontra no texto personalities.txt. 
Caso estas duas condições estejam certas introduz se o nome numa lista. 
Por último introduz se na base de dados a noticia com a lista de personalidades encontradas nesta.

\subsubsection{Sentiment Analysis}
A classe que implementa esta funcionalidade sentimentAnalisys / sentimentFlex.py.
A função sentiLexFlexToDict começa por obter toda a informação que se encontra no ficheiro SentiLex-flex-PT02.txt. 
E em seguida construir uma estrutura de dados mais fácil de aceder e com apenas a informação pretendida.
A função polarity recebe a estrutura de dados e a notícia que vai ser analisada. 
Esta função divide a notícia por frases. 
Depois em cada uma das frases é verificada a existência de expressões e palavras que se encontram na estrutura de dados recebida. 
Caso exista executa se uma operação de soma ou subtração de acordo com o valor da polaridade. 
Assim consegue se saber se uma certa notícia é positiva ou negativa.


\subsubsection{Super Search}
Esta funcionalidade trata três tipos de pesquisa: personalidades pesquisa de conteúdos e qual a personalidade mais ou menos popular
Quando se realize uma pesquisa de uma personalidade o resultado esperado é o nome da personalidade, as personalidades que se relacionam com a personalidade pesquisada e suas popularidades, a popularidade da personalidade pesquisada e as noticias onde a personalidade e referida.
A função que trata a descoberta de personalidades relacionadas é graphSearch.A função primeiro verifica se a personalidade introduzida para pesquisa se encontra no ficheiro
personalities.txt. Caso exista vai a base de dados procurar as noticias onde a personalidade e referida e obtém as outras personalidades que também foram referidas nessa noticia.
A função que indica a popularidade e personalityPopularity. A função primeiro verifica se a personalidade introduzida para pesquisa se encontra no ficheiro personalities.txt. Caso exista vai a base de dados procurar as noticias onde a personalidade e referida e verifica de que forma a personalidade é referida na notícia de forma positiva ou de forma negativa. No fim de percorrer todas as notícias alcançasse a popularidade da personalidade pesquisada.

Quando se realiza uma pesquisa por qual é a personalidade mais popular ou menos popular executa se ou a função whoIsMorePopular ou WhoIsMoreHated. 
Esta duas funções são executadas dependo da query que é introduzida caso seja introduzida “who is more popular” é executada a função whoIsMorePopular, caso seja introduzida “who is more hated” é executada a outra função. 
Estas duas funções recebem uma estrutura de dados que contem todas as personalidades e sua popularidade. 
Esta estrutura é criada pela função orderedByPopularity que obtém os nomes das personalidades e sua popularidade a partir da base de dados introduzindo as numa estrutura de dados que em seguida será ordenada usando a função sorted do python.
\section{Result analysis}

\subsection{Experimental Results}

\subsection{Measures of the results}


\section{Future Work}


\section{Conclusions}


\end{document}
